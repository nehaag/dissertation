\section{Application Aware Page Placement}
\label{binaryinstrument}
%\vspace{-0.05in}

Figure~\ref{fig:cdfannotation-bfs},~\ref{fig:cdfannotation-muumergpu},~\ref{fig:cdfannotation-needle} 
visually depicts what may be obvious to performance-conscious programmers:
\emph{certain data structures are accessed more frequently than others.} For
these programmers, the unbiased nature of BW-AWARE page placement is not
desirable because all data structures are treated equally.  This section
describes compiler tool-flow and runtime modifications that enable programmers
to intelligently steer specific allocations towards bandwidth- or
capacity-optimized memories and achieve near-oracle page placement performance.

To correctly annotate a program to enable intelligent memory placement decisions
across a range of systems, we need two pieces of information about a program:
(1) the relative memory access hotness of the data structures, and (2) the size
of the data structures allocated in the program.  To understand the importance
of these two factors, let us consider the following scenario. In a program there
are two data structure allocations with hotness H1 and H2\@.  If the
bandwidth-optimized memory capacity is large enough for BW-AWARE placement to be
used without running into capacity constraints, then BW-AWARE page placement
should be used irrespective of the hotness of the data structures. To make this
decision we must know the application runtime memory footprint. However, if the
application is capacity-constrained, then ideally the memory allocation from the
hotter data structure should be preferentially placed in the BO memory. In this
case, we need to know both the relative hotness and the size of the data
structures to optimally place pages.

\subsection{Profiling Data Structure Accesses in GPU Programs}
\label{profiler}
%While GPU programmers have a reputation for deep understanding of their
%application characteristics, as machines become more complex and programs rely
While expert programmers may have deep understanding of their application
characteristics, as machines become more complex and programs rely more on GPU
accelerated libraries, programmers will have a harder time maintaining this
intuition about program behavior.  To augment programmer intuition about memory
access behavior, we developed a new GPU profiler to provide information about
program memory access behavior.  

In this work we augmented {\tt nvcc} and {\tt ptxas}, NVIDIA's
production compiler tools for applications written in CUDA~\cite{cuda} (NVIDIA’s
explicitly parallel programming model), to support data structure access
profiling.  When profiling is enabled, our compiler's code generator emits
memory instrumentation code for all loads and stores that enables tracking of
the relative access counts to virtually addressed data structures within the
program.  As with the GNU profiler {\tt gprof}~\cite{GPROF}, the developer
enables a special compiler flag that instruments an application to perform
profiling.  The developer then runs the instrumented application on a set of
``representative'' workloads, which aggregates and dumps a profile of the
application.  

When implementing this GPU memory profiling tool, one of the
biggest challenges is that {\tt nvcc} essentially generates two
binaries: a host-side program, and a device-side program (that runs on
the GPU).  The profiler's instrumentation must track the execution of
both binaries.  On the host side, the profiler inserts code to track
all instances and variants of {\tt cudaMalloc}.  The instrumentation
associates the source code location of the {\tt cudaMalloc} with the
runtime virtual address range returned by it.  The host side code is
also instrumented to copy this mapping of line numbers and address
ranges to the GPU before each kernel launch.  The GPU-side
code is instrumented by inserting code before each memory operation
to check if the address falls within any of the
ranges specified in the map.  

For each address that falls within a
valid range, a counter associated with the range is incremented, using
atomic operations because the instrumentation code is highly
multi-threaded.  At kernel completion, this updated map is
returned to the host which launched the kernel to aggregate the
statistics about virtual memory location usage. Our profiler generates
informative data structure mapping plots, like those shown in

Figure~\ref{fig:cdfannotation-bfs},~\ref{fig:cdfannotation-muumergpu},~\ref{fig:cdfannotation-needle},
which application programmers can use to guide their understanding of the
relative access frequencies of their data structures, one of the two required
pieces of information to perform intelligent near-optimal placement within an
application.

\subsection{Memory Placement APIs for GPUs}
\label{api}

With a tool that provides programmers a profile of data structure hotness, 
they are armed with the information required to
make page placement annotations within their application, but they are
lacking a mechanism to make use of this information. To enable memory placement hints (which are 
not a functional requirement) for where data 
should be placed in a mixed BO-CO memory system, we also provide an alternate method 
for allocating memory. 
We introduce an additional argument 
to the {\tt cudaMalloc} memory allocation functions that specifies in which domain the memory should be
allocated (BO or CO) or to use BW-AWARE placement (BW). For example:\\
\vspace{0.1in}
\hspace{0.25in}{\color{black}{\small {\tt{cudaMalloc(void **devPtr, size\_t size, enum hint)}}}}

This hint is not machine specific and simply indicates if the CUDA memory allocator
should make a best effort attempt to place memory within a BO or CO optimized memory
using the underlying OS libNUMA functionality or fall back to the bandwidth-aware allocator. 
By providing an abstract hint, the CUDA runtime, rather than the programmer, becomes responsible
for identifying and classifying the machine topology of memories as bandwidth or capacity
optimized.  While we have assumed bandwidth information is available in our proposed system bandwidth 
information table, programmatic discovery of memory zone bandwidth is also possible as a fall back
mechanism~\cite{McCalpin1995}. In our implementation, memory hints are honored unless the memory pool is 
filled to capacity, in which case the allocator will fall back to the alternate domain. 
If no placement hint is provided, the CUDA runtime will fall back to using the application 
agnostic BW-AWARE placement for unannotated memory allocations.
{\color{black} When a hint is supplied, the {\tt cudaMalloc} routine uses the
{\tt mbind} system call in Linux to perform placement of the data structure in
the corresponding memory.}
%OS support for allocating memory
%from specific memory zones is already present, e.g., in the form of the {\tt
%mbind} system call in Linux. Thus, this mechanism does not require any
%modifications to the OS.}

\subsection{Program Annotation For Data Placement}

Our toolkit now includes a tool for memory profile generation and a mechanism to specify
abstract memory placement hints. While programmers may choose to use this information
directly, optimizing for specific machines, making these hints performance portable across a range of machines
is harder as proper placement depends on application footprint as well as the memory
capacities of the machine. For performance portability,
the hotness and allocation size information must be annotated in the program before any heap allocations occur.
We enable annotation of this information as two arrays of values
that are linearly correlated with the order of the memory allocations in the program.
For example Figure~\ref{fig:code} shows the process of hoisting the size
allocations manually into the {\tt size} array and hotness into the {\tt hotness} array. 

\lstset { %
    language=C++,
    basicstyle=\footnotesize,% basic font setting
    captionpos=b,
    frame=top,frame=bottom
}
\begin{filecontents*}{code.txt}
// n:input parameter
cudaMalloc(devPtr1, n*sizeof(int));
cudaMalloc(devPtr2, n*n);
cudaMalloc(devPtr3, 1000);
// n: input parameter
// size[i]: Size of data structures
// hotness[i]: Hotness of data structures
size[0] = n*sizeof(int);
size[1] = n*n;
size[2] = 1000;
hotness[0] = 2;
hotness[1] = 3;
hotness[2] = 1;

// hint[i]: Computed data structure placement hints
hint[] = GetAllocation(size[], hotness[]);
cudaMalloc(devPtr1, size[0], hint[0]);
cudaMalloc(devPtr2, size[1], hint[1]);
cudaMalloc(devPtr3, size[2], hint[2]);
\end{filecontents*}

\begin{figure}[t]
    \subfloat[Original code dependent allocations]{\lstinputlisting[lastline=4]{code.txt}}\\
    \hfill
    \subfloat[Final code]{\lstinputlisting[firstline=5,
lastline=20]{code.txt}\label{c}}
    \caption{Annotated pseudo-code to do page placement at runtime taking into account
relative hotness of data structures and data structure sizes}
    \label{fig:code}
\end{figure}

We provide a new runtime function {\tt GetAllocation} that then uses these two
pieces of information, along with the discovered machine bandwidth topology, to
compute and provide a memory placement hint to each allocation. {\tt
GetAllocation} determines appropriate memory placement hints by computing the
ideal (BO or CO) memory location by first calculating the cumulative footprint
of all data structures and then calculating the total number of identified data
structures from {\tt[1:N]} that will fit within the bandwidth-optimized memory
before it exhausts the BO capacity.

It is not critical that programmers provide annotations for all memory
allocations, only large or performance critical ones. {\color{black}For applications that make
heavy use of libraries or dynamic decisions about runtime allocations, it may not be
possible to provide good hinting decisions because determining the size of data structures
allocated within libraries calls is difficult, if not impossible, in many cases.}.
While this process may seem impractical to a traditional high level language
programmer, examining a broad range of GPU compute workloads has shown that in
almost all GPU-optimized programs, the memory allocation calls are already
hoisted to the beginning of the GPU compute kernel. The CUDA C Best Practices
Guide advises the programmer to minimize memory allocation and
de-allocation in order to achieve the best performance~\cite{CUDAGuide}.
