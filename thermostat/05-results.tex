\section{Results}
\label{results}

\subsection{Thermostat Performance Evaluation}
All-DRAM with 4KB\\
All-DRAM with THP\\
2-level memory at 4KB with kstaled doing page placement\\
2-level memory with THP with kstaled doing page placement\\ 
2-level memory with our design\\

\subsection{DRAM Cost Analysis}

Once the Thermostat controller has been built, we propose to perform the
following experiments. First, we will study the impact on application
performance of employing slow-mem on several cloud-computing
applications~\cite{Perfkitbenchmarker}. Satisfying SLAs is compulsory for these
applications as it directly impacts businesses they are deployed in. We will
design our mechanism so that SLAs are not violated in all of these benchmarks.

Second, we will perform a sensitivity study of different policies that Thermostat
controller employs and estimate the corresponding TCO savings. The goal of
employing Thermostat controller is to reduce main memory TCO. However, Thermostat is
expected to increase the CPU usage as well, which will add to the TCO cost. We
will investigate this trade-off of plausible main memory TCO reduction and extra
CPU usage cost.

Finally, we will do a trade-off analysis of performance and TCO with Thermostat.
Looking at only performance (throughput or latency) or TCO can lead one to wrong
conclusions. We will analyze all possible potential impacts of Thermostat on TCO,
including, e.g., cost and power consumption of new memory technologies, power
consumption of Thermostat controller on CPU, additional provisioning that may be
necessary because of lower throughput at SLA etc. One of our end goals is
to come up with a detailed analysis of when using such memory technologies can
be beneficial and drive the adoption of such technologies by internet giants.

\subsection{Discussion}
Varibale that this design will influence in other parts of the system, CPU,
power analysis
