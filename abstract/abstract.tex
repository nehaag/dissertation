%Flow of abstract
%Why hetereogenous memory systems
%What are the challenges in managing such systems
%How did we address them
%Insight into our proposal
%What are the big/key takeaways

Systems from smartphones to data-centers to supercomputers are increasingly
heterogeneous, comprising various memory technologies and core types.
Heterogeneous memory systems provide an opportunity to suitably match varying
memory access patterns in applications, reducing CPU time and thus the cost to
run the application.  CPU time savings increase performance per dollar resulting
in aggregate savings of millions of dollars in large-scale systems. However,
with increased provisioning of main memory capacity per machine and differences
in memory characteristics (for example, bandwidth, latency, cost, and density),
memory management in such heterogeneous memory systems poses multi-fold
challenges on system programmability and design. Decisions on data placement and
movement have to be made at runtime – in an application-transparent manner –
while suitably matching application memory access patterns so as to utilize the
heterogeneous resources efficiently.

In this thesis, we tackle memory management of two heterogeneous memory systems:
(a) CPU-GPU systems with a unified virtual address space, and (b) Cloud
computing platforms that can deploy cheaper but slower memory technologies
alongside DRAMs. We discuss the challenges presented by upcoming heterogeneous
CPU-GPU memory systems and propose solutions to enable their adoption by
simplifying the programming model and design effort needed to build such complex
systems. We show that current page placement policies are not sufficient to
maximize GPU performance in heterogeneous CPU-GPU memory systems and propose an
application agnostic Bandwidth-Aware (BW- AWARE) placement policy that maximizes
GPU throughput.
%by balancing page placement
%across the memories based on the aggregate memory bandwidth available in a
%system.
%Our results show that BW-AWARE placement outperforms the existing Linux
%INTERLEAVE and LOCAL policies by 35\% and 18\% on average for GPU compute
%workloads.  We build upon BW-AWARE placement by developing a compiler-based
%profiling mechanism that provides programmers with information about GPU
%application data structure access patterns.  Combining this information with
%simple program-annotated hints about memory placement, our hint-based page
%placement approach performs within 90\% of oracular page placement on average,
%largely mitigating the need for costly dynamic page tracking and migration.
%
%Historically, GPU-based HPC applications have had a substantial memory
%bandwidth advantage over CPU-based workloads due to using GDDR rather than DDR
%memory.  However, past GPUs required a restricted programming model where
%application data was allocated up front and explicitly copied into GPU memory
%before launching a GPU kernel by the programmer. Recently, GPUs have eased this
%requirement and now can employ on-demand software page migration between CPU
%and GPU memory to obviate explicit copying.  In the near future, CC-NUMA
%GPU-CPU systems will appear where software page migration is an optional choice
%and hardware cache-coherence can also support the GPU accessing CPU memory
%directly.
To enhance programmability we explore dynamic page migration policies that do
not require any programmer annotation.
%We describe the trade-offs and considerations in relying on hardware
%cache-coherence mechanisms versus using software page migration to optimize the
%performance of memory-intensive GPU workloads.
%We show that page migration decisions based on page access frequency
%alone are a poor solution and that a broader solution exploiting locality in virtual
%address-based program
%%to enable aggressive memory prefetching
%%combined with bandwidth balancing
%is desirable to maximize performance.
%We present a software runtime system requiring minimal hardware support that,
%on average, outperforms CC-NUMA-based accesses by 1.95$\times$, performs 6\%
%better than the legacy CPU to GPU {\tt memcpy} regime by intelligently using
%both CPU and GPU memory bandwidth, and comes within 28\% of oracular page
%placement, all while maintaining the relaxed memory semantics of modern GPUs.
%
To simplify implementing hardware cache
coherence in heterogeneous CPU-GPU systems,
%Cache coherence is ubiquitous in shared memory multi\hyp{}processors because it
%provides a simple, high performance memory abstraction to programmers.  Recent
%work suggests extending hardware cache coherence between CPUs and GPUs to help
%support programming models with tightly coordinated sharing between CPU and GPU
%threads.  However, implementing hardware cache coherence is particularly
%challenging in systems with discrete CPUs and GPUs that may not be produced by
%a single vendor. Instead,
we propose selective caching, wherein we disallow GPU caching of any memory that
would require coherence updates to propagate between the CPU and GPU, thereby
decoupling the GPU from vendor-specific CPU coherence protocols.
%We propose several architectural improvements to offset the performance penalty
%of selective caching including aggressive request coalescing, CPU-side coherent
%caching for GPU-uncacheable requests, and CPU--GPU interconnect optimizations.
%Moreover, current GPU workloads access many read-only memory pages; we exploit
%this property to allow promiscuous GPU caching of these pages, relying on
%page-level protection, rather than hardware cache coherence, to ensure
%correctness.   These optimizations bring a selective caching GPU implementation
%to within 93\% of a hardware cache-coherent implementation without the need to
%integrate CPUs and GPUs under a single hardware coherence protocol.

Heterogeneous memories are making inroads in large scale data-centers with the
advent of denser/cheaper memory technologies, such as Intel/Micron’s 3D XPoint.
We make a case for a two-tiered memory system where cheaper (per bit) memories
will be deployed alongside DRAM. Such a heterogeneous memory system can enable
reduced DRAM provisioning per application by shifting infrequently accessed cold
data to slow memory, reducing spending on main memory in large-scale systems.
Past research on two-tiered main memory has assumed a 4KB page size.  However,
we demonstrate that 2MB transparent huge pages are performance critical in
cloud applications especially in virtualized cloud environments.  To enable such
deployments, we present Thermostat, an application-transparent huge-page-aware
mechanism to place pages in a dual-technology hybrid memory system while
achieving both the cost advantages of two-tiered memory and performance
advantages of transparent huge pages.  We present an online page classification
mechanism that accurately classifies both 4KB and 2MB pages as hot or cold while
incurring no observable performance overhead across several representative cloud
applications. We implement Thermostat in Linux kernel version 4.5 and evaluate
its effectiveness on representative cloud computing workloads running under KVM
virtualization. We show that Thermostat migrates up to 50\% of application
footprint to slow memory while limiting performance degradation to 3\%, thereby
reducing memory cost.
%We propose to develop a transparent huge-page-aware two-tiered memory solution,
%targeting virtualized cloud applications, which integrates support for dynamic
%page migration and transparent huge pages, achieving both the capacity/cost
%advantages of two-tiered memory and performance advantages of huge pages. Hot
%regions within otherwise cold huge pages present a central challenge to our
%objective. We propose translation facades, a 4KB translation that remaps a
%portion of a 2MB mapping with an alternate physical address or permissions, to
%facilitate remapping hot portions of cold huge pages.
