Current OS page placement policies are optimized for both homogeneous memory and
latency sensitive systems. With the emergenece of shared virtual address CPU-GPU
systems and new memory technologies like Intel 3D XPoint memory management
policies need to account for difference in bandwidth, coherence domains, cost
per dollar of different memory technologies, deployed to use concurrently by the
computing agents.

In this thesis we have examined a pressing problem that the GPU industry is
facing on how to best handle memory placement for upcoming cache coherent
GPU-CPU systems.  While the problem of page placement in heterogeneous memories
has been examined extensively in the context of CPU-only systems, the
integration of GPUs and CPUs provides several unique challenges.  First, GPUs
are extremely sensitive to memory bandwidth, whereas traditional memory
placement decisions for CPU-only systems have tried to optimize latency as their
first-order concern.  Second, while traditional SMP workloads have the option to
migrate the executing computation between identical CPUs, mixed GPU-CPU
workloads do not generally have that option since the workloads (and programming
models) typically dictate the type of core on which to run.  
%This leaves data migration as the only option for co-locating data and
%processing resources.
Finally, to support increasingly general purpose programming models, where the
data the GPU shares a common address space with the CPU and is not necessarily
known before the GPU kernel launch, programmer-specified up-front data migration
is unlikely to be a viable solution in the future.

We propose a new BW-AWARE page placement policy that uses memory system
information about heterogeneous CPU-GPU memory system characteristics to place
data appropriately, achieving 35\% performance improvement on average over
existing policies without requiring any application awareness.
%In future CC-NUMA systems, BW-AWARE placement improves the performance optimal
%capacity by better using all system resources.  But some applications may wish
%to size their problems based on total capacity rather than performance. In such
%cases, In cases where application does not entirely fit the desirable
%performance optimal capacity, we provide insight into how to optimize data
%placement by using the CDF of the application in combination with application
%annotations enabling intelligent runtime controlled page placement decisions.
%We propose a profile-driven application annotation scheme that enables improved
%placement without requiring any runtime page migration. While only the
%beginning of a fully automated optimization system for memory placement, we
%believe that the performance gap between the current best OS INTERLEAVE policy
%and the annotated performance (min 1\%, avg 20\%, max 2x) is enough that
%further work in this area is warranted as mobile, desktop, and HPC memory
%systems all move towards mixed CPU-GPU CC-NUMA heterogeneous memory systems.
%
%We have presented a solution to a limited-scope data placement problem for
%upcoming GPU-CPU systems to enable intelligent migration of data into high
%bandwidth GPU-attached memory.  
%
We also present a intelligent dynamic page migration solution that maximizes the
bandwdith utilization of different memory technologies in heterogeneous CPU-GPU
memory system.  We identify that demand-based migration alone is unlikely to be
a viable solution due to both application variability and the need for
aggressive prefetching of pages the GPU is likely to touch, but has not touched
yet.  The use of range expansion based on virtual address space locality, rather
than physical page counters, provides a simple method for exposing application
locality while eliminating the need for hardware counters exploiting the memory
access pattern of GPU computer applications.  Developing a system with minimal
hardware support is important in the context of upcoming CPU-GPU systems, where
multiple vendors may be supplying components in such a system and relying on
specific hardware support on either the GPU or CPU to achieve performant page
migration may not be feasible.  Our migration solution is able to outperform
CC-NUMA access alone by 1.95$\times$, legacy application {\tt memcpy} data
transfer by 6\%, and come within 28\% of oracular page placement.

These memory migration policies optimize the performance of GPU workloads with
little regard for CPU performance.  We have shown that intelligent use of the
high bandwidth memory on the GPU can account for as much as a 5-fold performance
increase over traditional DDR memory systems.  While this is appropriate for
applications where GPU performance dominates Amdahl's optimization space,
applications with greater data sharing between the CPU and GPU are likely to
evolve.  Understanding what these sharing patterns look like and
balancing the needs of a latency-sensitive CPU versus a bandwidth-hungry GPU is
an open problem. Additionally, with memory capacities growing ever larger and
huge pages becoming more commonly used, evaluating the trade-off between
reducing TLB shootdowns and longer page copy times will be necessary to maintain
the high memory bandwidth critical for good GPU performance.

Introducing globally visible shared memory in future CPU/GPU systems improves
programmer productivity and significantly reduces the barrier to entry of using
such systems for many applications.  Hardware cache coherence can provide such
shared memory and extend the benefits of on-chip caching to all memory within
the system.  However, extending hardware cache coherence throughout the GPU
places enormous scalability demands on the coherence implementation.  Moreover,
integrating discrete processors, possibly designed by distinct vendors, into a
single coherence protocol is a prohibitive engineering and verification
challenge.  

In this thesis, we demonstrate that CPUs and GPUs do not need to be hardware
cache-coherent to achieve the simultaneous goals of unified shared memory and
high GPU performance.  Our results show that \textit{selective caching} with
request coalescing, a CPU-side GPU client cache, variable-sized transfer units
can perform within 93\% of a cache-coherent GPU for applications that do not
perform fine grained CPU--GPU data sharing and synchronization. We also show
that promiscuous read-only caching benefits memory latency sensitive
applications using OS page-protection mechanisms rather than relying on hardware
cache coherence.  Selective caching does not needlessly force hardware cache
coherence into the GPU memory system, allowing decoupled designs that can
maximize CPU and GPU performance, while still maintaining the CPU's traditional
view of the memory system.
