\vspace{-.05in}
\section{Methodology}
\label{methodology}

We evaluate selective caching via simulation on a system containing discrete
CPUs and GPUs with DDR4 and GDDR5 memories attached to the CPU and GPU,
respectively.  We discuss our baseline simulation environment in
Chapter~\ref{chap:asplos2015:baseline-methodology}.  To implement various architectural components
we modify our framework with simulation parameters shown in
Table~\ref{tab:sc-methodology}.  We use bandwidth-aware page placement for all
simulations as it has been shown to be the best page placement strategy without
requiring any application profiling or program modification~\cite{Agarwal2015}.
In our simulated system, this page placement results in 20\% of the GPU workload
data being placed within the CPU-attached memory with 80\% residing in the
GPU-attached memory.  

In our system, the CPU is connected to the GPU via a full duplex CPU--GPU
interconnect. The interconnect has peak bandwidth of 90GB/s using 16B flits for
both data and control messages with each data payload of up to 128B requiring a
single header flit.  Thus, for example, a 32B data message will require sending
1 header flit + 2 data flits = 3 flits in total.
%To simulate an additional interconnect hop to remote CPU memory, we model an
%additional fixed, pessimistic, 100 cycle interconnect latency to access the
%DDR4 memory from the GPU\@. This overhead is derived from the single additional
%interconnect hop latency found in SMP CPU-only designs, such as the Intel
%Xeon~\cite{INTELXEONE5V3}.
When simulating request coalescing within the GPU, we use the same number of
MSHRs as the baseline configuration but allow the MSHRs to have their own local
return value storage in the cacheless request coalescing case.  The CPU-side GPU
client cache is modeled as an 8-way set associative, write-through, no
write-allocate cache with 128B line size of varying capacities shown later in
Section~\ref{mccache}. The client cache latency is 200 cycles, comprising 100
cycles of interconnect and 100 cycles of cache access latency.  To support
synchronization operations between CPU and GPU, we augment the GPU MSHRs to
support atomic operations to data homed in either physical memory; we assume the
CPU similarly can issue atomic accesses to either memory.

\begin{table}[t]
\begin{center}
\begin{tabular}{|l|l|}
%\hline
%Simulator & GPGPU-Sim 3.x\\
%\hline
%GPU Arch & NVIDIA GTX-480 Fermi-like\\
%\hline
%GPU Cores& 15 SMs @ 1.4Ghz\\
%\hline
%L1 Caches & 16kB/SM, 3 cycle latency\\
%\hline
%L1 MSHRs & 64 Entries/L1\\
%\hline
%L2 Caches & 128kB/Channel, 120 cycle lat.\\
%\hline
%L2 MSHRs & 128 Entries/L2 Slice\\
%\hline
\hline
\multicolumn{2}{|c|}{Memory System}\\
\hline
CPU Client Cache & 512KB, 200 cycle latency\\
\hline
GPU GDDR5 & 8-channels, 336GB/sec aggregate\\
\hline
CPU DDR4& 4-channels, 80GB/sec aggregate\\
\hline
SW Page Faults& 16 concurrent per SM\\
\hline
DRAM Timings & \multicolumn{1}{|l|}{RCD=RP=12, RC=40, CL=WR=12}\\
\hline
DDR4 Burst Len.& 8\\
\hline
\hline
\multicolumn{2}{|c|}{CPU--GPU Interconnect}\\
\hline
Link Latency& 100 GPU core cycles\\
\hline
Link Bandwidth& 90 GB/s Full-Duplex\\
\hline
Req. Efficiency& 32B=66\%, 64B=80\%, 128B=88\%\\
\hline
\end{tabular}
\caption{Parameters for experimental GPGPU based simulation environment.}
\label{tab:sc-methodology}
\end{center}
\vspace{-.1in}
\end{table}

To model promiscuous read-only caching, we initially mark all the pages (4kB in
our system) in DDR as read-only upon GPU kernel launch. When the first write is
issued to each DDR page, the ensuing protection fault invalidates the TLB entry
for the page at the GPU.  When the faulting memory operation is replayed, the
updated PTE is loaded, indicating that the page is uncacheable.  Subsequent
accesses to the page are issued over the CPU--GPU interconnect.  Pages marked
read-write are never re-marked read-only during GPU kernel execution. Using the
page placement policy described earlier in this section, the GPU is able to
cache 80\% of the application footprint residing in GPU memory. We vary our
assumption for the remote protection fault latency from 20-40us and assume
support for up to 16 pending software page protection faults per SM; a
seventeenth fault blocks the SM from making forward progress on any warp.

We evaluate results using the Rodinia and United States Department of Energy
benchmark suites. We execute the applications under the CUDA 6.0 weak
consistency memory model.  While we did evaluate workloads from the
Parboil~\cite{Parboil} suite, we found that these applications have
uncharacteristically high cache miss rates, hence even in the hardware
cache-coherent case, most memory accesses go to the DRAM. As such, we have
chosen to omit these results because they would unfairly indicate that selective
caching is performance equivalent to a theoretical hardware cache-coherent GPU.
In Section~\ref{results} we report GPU performance as application throughput,
which is inversely proportional to workload execution time.
