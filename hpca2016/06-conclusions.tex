\section{Conclusion}
%Introducing globally visible shared memory in future CPU/GPU systems
%improves programmer productivity and significantly reduces the barrier
%to entry of using such systems for many applications. 
%Hardware cache coherence can provide such shared memory and
%extend the benefits of on-chip caching to all memory within the system.
%\ignore{Hardware cache coherence in future CPU/GPU systems would allow the integration
%of the separate CPU and GPU programming paradigms under a single
%uniform model, leveraging the benefits of on-chip caching for all
%memory within the system.}  However, extending hardware cache coherence 
%throughout the GPU places enormous
%scalability demands on the coherence implementation.  Moreover, integrating
%discrete processors, possibly designed by distinct vendors,
%into a single coherence protocol is a prohibitive engineering and
%verification challenge.  

In this chapter, we demonstrate that CPUs and GPUs do not need to be hardware
cache-coherent to achieve the simultaneous goals of unified shared memory and
high GPU performance.  Our results show that \textit{selective caching} with
request coalescing, a CPU-side GPU client cache, variable-sized transfer units
can perform within 93\% of a cache-coherent GPU for applications that do not
perform fine grained CPU--GPU data sharing and synchronization. We also show
that promiscuous read-only caching benefits memory latency sensitive
applications using OS page-protection mechanisms rather than relying on hardware
cache coherence.  Selective caching does not needlessly force hardware cache
coherence into the GPU memory system, allowing decoupled designs that can
maximize CPU and GPU performance, while still maintaining the CPU's traditional
view of\ignore{ a hardware coherent} the memory system.
