\begin{abstract}
Historically, GPU-based HPC applications have had a substantial memory bandwidth
advantage over CPU-based workloads due to using GDDR rather than
DDR memory.  However, past GPUs required a restricted programming
model where application data was allocated up front and explicitly copied into
GPU memory before launching a GPU kernel by the programmer. Recently, GPUs have eased this 
requirement and now can employ on-demand software page migration between CPU and 
GPU memory to obviate explicit copying.  In the near future, CC-NUMA GPU-CPU systems will 
appear where software page migration is an optional choice and
hardware cache-coherence can also support the GPU accessing CPU memory directly.  
In this work, we describe the trade-offs and considerations in relying on 
hardware cache-coherence mechanisms versus using software page migration to optimize the performance 
of memory-intensive GPU workloads.  We show that page migration decisions based on page
access frequency alone are a poor solution and that a broader solution using virtual
address-based program locality to enable aggressive memory prefetching combined with
bandwidth balancing is required to maximize performance.  We present a software runtime
system requiring minimal hardware support that, on average, outperforms CC-NUMA-based
accesses by 1.95$\times$, performs 6\% better than the legacy CPU to GPU {\tt
memcpy} regime by 
intelligently using both CPU and GPU memory bandwidth, and comes within 28\% of oracular page placement,
all while maintaining the relaxed memory semantics of modern GPUs.
\end{abstract}
