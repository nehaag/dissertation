\section{Overall framework for Heterogeneous CPU-GPU Systems}
To evaluate BW-AWARE page placement, we simulated a heterogeneous memory system
attached to a GPU comprised of both bandwidth-optimized GDDR and
cost/capacity-optimized DDR where the GDDR memory is attached directly to the
GPU\@.  {\color{black} No contemporary GPU system is available which supports
cache-coherent access to heterogeneous memories.  Commonly available
PCIe-attached GPUs are constrained by interconnect bandwidth and lack of
cache-coherence; while cache-coherent GPU systems, such as AMD's Kaveri, do not
ship with heterogeneous memory. 

Our simulation environment is based on GPGPU-Sim~\cite{gpgpusimIspass09} which
has been validated against NVIDIA's Fermi class GPUs and is reported to match
hardware performance with up to 98.3\% accuracy~\cite{gpgpusimManual}}.  We
modified GPGPU-Sim to support a heterogeneous GDDR5-DDR4 memory system with the
simulation parameters listed in Table~\ref{tab:methodology-basic}.  We made several
changes to the baseline GTX-480 model to bring our configuration in-line with
the resources available in more modern GPUs, including a larger number of MSHRs
and higher clock frequency.

As noted in Section~\ref{heterogeneous_background}, attaching the
capacity-optimized memory directly to the GPU is functionally equivalent to
remote CPU attached memory, but with different latency parameters.  To simulate
an additional interconnect hop to remote CPU-attached memory, we model a fixed,
pessimistic, additional 100 cycle latency to access the DDR4 memory from the
GPU\@. This overhead is derived from the single additional interconnect hop
latency found in SMP CPU-only designs such as the Intel XEON~\cite{INTELXEON}\@.
Our heterogeneous memory model contains the same number of MSHRs per memory
channel as the baseline memory configuration.  The number of MSHRs in the
baseline configuration is sufficient to effectively hide the additional
interconnect latency to the DDR memory in Figure~\ref{fig:latency}. Should MSHR
quantity become an issue when supporting two level memories, previous work has
shown that several techniques can efficiently increase MSHRs with only modest
cost~\cite{ref:tuck:scalablemisshandling, ref:minikin:prefetch}.

\begin{table}[t]
\begin{center}
%\begin{small}
\begin{tabular}{|l|l|}
\hline
Simulator & GPGPU-Sim 3.x\\
\hline
GPU Arch & NVIDIA GTX-480 Fermi-like\\
\hline
GPU Cores& 15 {\color{black}SMs} @ 1.4Ghz\\
\hline
L1 Caches & 16kB/SM, 3 cycle latecy \\
\hline
L1 MSHRs & 64 Entries/L1\\
\hline
L2 Caches & 128kB/Channel, 120 cycle lat.\\
\hline
L2 MSHRs & 128 Entries/L2 Slice\\
\hline
%\hline
%\multicolumn{2}{|c|}{Memory system}\\
%\hline
%GPU-Local GDDR5 & 8-channels, 200GB/sec aggregate\\
%\hline
%GPU-Remote DDR4& 4-channels, 80GB/sec aggregate\\
%\hline
%DRAM Timings & \multicolumn{1}{|l|}{RCD=RP=12,RC=40,CL=WR=12}\\
%\hline
%GPU-CPU &  100 GPU core cycles\\
%Interconnect Latency & \\
%\hline
\end{tabular}
\caption{Simulation environment for heterogeneous CPU-GPU memory system.}
\label{tab:methodology-basic}
%\end{small}
\end{center}
\vspace{-0.15in}
\end{table}
